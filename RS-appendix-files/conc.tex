\subsection{Making a decision}

We conclude by observing that these models may be tuned to target specific network scenarios and requirements. One network may require one set of Reed-Solomon parameters, while a different network may require another. In general, the closer $m/n$ is to 1, the more rebuilds per month should be expected under a fixed churn rate. While having a larger ratio for $m/n$ increases file durability for any given churn rate, it comes at the expense of more bandwidth used since repairs are triggered more often. To maintain a low mean rebuilds/month value while also maintaining a higher file durability, the aim should be to increase the value of $n$ as much as feasible given other network conditions (latency, download speed, etc.), which allows for a lower relative value of $r$ while still not jeopardizing file durability.

Informally, it takes longer to lose more pieces under a given fixed network size and churn rate. Therefore, to maximize durability while minimizing repair bandwidth usage, $n$ should be as large as existing network conditions allow. This allows for a value of $m$ that is relatively closer to $k$, reducing the mean rebuilds/month value, which in turn lowers the amount of repair bandwidth used.

For example, assume we have a network with a mean time to failure of six months.
Suppose we consider the same file encoded with two different RS parameters:
one under a $(20,40)$ schema and the other as an $(30,80)$ schema. If we set $m$ so that $m=k+10$ for both cases, we observe from the above table
that the bandwidth repair ratio is is $0.87$ in the $(20,40)$ case and is $0.60$ in the $(40,80)$ case. Both encoding schemes have similar durability, as a repair in both cases is triggered when there are $k+10$ pieces left; even though, the mean rebuilds per month
is empirically and theoretically lower for the $(40,80)$ case using $m=k+10$.
