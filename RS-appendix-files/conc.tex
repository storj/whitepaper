\subsection{Making a decision}

We conclude by observing that these models may be tuned to target specific network scenarios and requirements. One network may require one set of Reed-Solomon parameters, while a different network may require another. In general, the closer $m/n$ is to 1, the more rebuilds per month one should expect under a fixed churn rate. While having a larger ratio for $m/n$ increases file durability for any given churn rate, it comes at the expense of more bandwidth used since repairs are triggered more often. To maintain a low mean rebuilds/month value while also maintaining a higher file durability, one may aim to increase the value of $n$ as much as feasible given other network conditions (latency, download speed, etc.), which allows for a lower relative value of $r$ while still not jeopardizing file durability. 

Informally, it takes longer to lose more pieces under a given fixed network size and churn rate, so to maximize durability while minimizing repair bandwidth usage, $n$ should be as large as existing network conditions allow for. This allows for a value of $m$ that is relatively closer to $k$, reducing the mean rebuilds/month value, which in turn lowers the amount of repair bandwidth used. 

For example, assume we have a fixed network size of $10,000$ nodes and a fixed churn rate of 0.1, so that a fixed number of $1,000$ pieces are lost on the network each month\footnote{
Technically, we assume this figure represents ``churn'', so while we assume $1,000$ nodes are lost each month, we also assume $1,000$ new nodes have come online.
}. 
Suppose we consider the same file encoded with two different RS parameters: once under a $(40:20)$ schema and the other as an $(80:40)$ schema. If we want to set $m$ so that $m=k+10$ for both cases, we observe from Tables \ref{table:n=40} and \ref{table:n=80} that the expected mean rebuilds is $0.416$ in the $(40:20)$ case and is $0.292$ in the $(80:40)$ case. Both encoding schemes have similar durabilities, as a repair in both cases is triggered when there are $k+10$ pieces left, though the mean rebuilds per month is empirically and theoretically lower for the $(80:40)$ case using $m=k+10$. 

Finally, we remark that this work has focused on analyzing the bandwidth used by different RS encoding schemes under differing network conditions. As a prospect for future research, we propose considering how the choice of $m$ relative to $k$ affects file durability under differing network conditions. We expect that the choice of $m$ relative to $k$ depends on the ratio of $k/C$, where $C$ is the total number of nodes on the network, and on the churn rate $c$. If the number of required pieces $k$ is large relative to $C$ and $c$, the expected number of shards lost per interval is increased, which must be taken into account when selecting $m$ to ensure that the file is not lost.