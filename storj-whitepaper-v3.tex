\documentclass[a4paper,10pt]{article}
\usepackage[hidelinks]{hyperref}
\def\UrlBreaks{\do\/\do-} % breaks long url in references
\usepackage{graphicx}
\usepackage[english]{babel}
\usepackage{listings}
\usepackage[labelfont=it,
  textfont={it},singlelinecheck=on,justification=centering]{caption}
\usepackage{amsmath}
\usepackage{float}
\usepackage{xcolor}

\lstset{basicstyle=\ttfamily\footnotesize,breaklines=true}
\lstset{numbers=left, numberstyle=\tiny, stepnumber=1, numbersep=5pt}
\lstset{language=TeX}
\setlength{\parskip}{1em}
\renewcommand{\baselinestretch}{1.2}
\newcommand{\x}[1]{{\tt #1}}
\newcommand{\code}[1]{{\tt #1}}
\newcommand{\todo}[1]{{\color{red} TODO #1}}

\title{\textbf{Storj\\A Peer-to-Peer Cloud Storage Network}}
\author{\\
\parbox{\linewidth}{\centering\small
Alex Bender (bender@storj.io),
Alex Leitner (alex@storj.io),
Benjamin Sirb (bens@storj.io),
Braydon Fuller (braydon@storj.io),
Bryan White (bryan@storj.io),
Chris Pollard (cpollard1001@gmail.com),
Dennis Coyle (dennis@storj.io),
Dylan Lott (dylan@storj.io),
Garrett Ransom (garrett@storj.io),
Gordon Hall (gordonhall@openmailbox.org),
James Hagans (jhagans@storj.io),
James Prestwich (james@storj.io),
John Gleeson (jg@storj.io),
Josh Brandoff (josh.brandoff@gmail.com),
JT Olio (jt@storj.io),
Kaloyan Raev (kaloyan@storj.io),
Kishore Aligeti (kishore@storj.io),
Nadine Farah (nadine@storj.io),
Nat Villasana (nat@storj.io),
Patrick Gerbes (patrick@storj.io),
Philip Hutchins (phillip@storj.io),
Shawn Wilkinson (shawn@storj.io),
Tome Boshevski (tome@storj.io)}\\
\\
\small With contributions from: Vitalik Buterin (v@buterin.com)
}
\date {April 26, 2018 \\ v3.0}

\begin{document}
\maketitle

\begin{abstract}
A peer-to-peer cloud storage network implementing client-side encryption would
allow users to transfer and share data without reliance on a third party storage
provider. The removal of central controls would mitigate most traditional data
failures and outages, as well as significantly increase security, privacy, and
data control. In the past, peer-to-peer networks have generally been infeasible
for production storage systems, as data availability is a function of
popularity, rather than utility. We propose a solution in the form of a
challenge-response verification system coupled with direct payments. In this
way we can periodically check data integrity and offer rewards to peers
maintaining data. We further propose a model for addressing access and
performance concerns with a set of independent or federated nodes.
\end{abstract}

\section{Introduction}

Cloud storage has come to rely almost exclusively on large storage providers
acting as trusted third parties to transfer and store data. This system suffers
from the inherent weaknesses of a trust-based model. Because client-side
encryption is nonstandard, the traditional cloud is often vulnerable to a
variety of security threats, including man-in-the-middle attacks, malware, and
application flaws that expose private consumer and corporate data. Moreover,
because many storage devices rely on the same infrastructure, failures are
correlated across files and systems.

A decentralized cloud storage network offers many advantages compared to
datacenter-based cloud storage. Data security can be maintained using
client-side encryption, while data integrity will be maintained via a proof of
retrievability. The impact of infrastructure failures and security breaches will
be greatly reduced. An open market for data storage may drive down costs for
various storage services by enabling more parties to compete using existing
devices. Data on the network will be resistant to censorship, tampering,
unauthorized access, and data failures. This paper describes a concrete
implementation of such a network, and a set of tools for interacting with that
network.

\section{Design constraints}

Before designing a system, it's important to understand the requirements of
said system, so we'll begin with a discussion of Storj's design constraints.

\subsection{S3 compatibility}

The flagship cloud storage product is Amazon's Simple Storage Service, or S3
for short. Most cloud storage products provide some form of compatibility with
the S3 API.

Until a decentralized cloud storage protocol is the {\em lingua franca} of
storage protocols, a graceful transition must be allowed for users with data
currently on a centralized provider who are interested in the benefits of
decentralized cloud storage but have low tolerance for switching costs.

For Storj to compete successfully in the wider cloud storage industry and bring
decentralized cloud storage to the mainstream, thus enabling more people
greater security and less centralized control, applications built against S3
should be able to be made to work with Storj with minimal friction and changes.
This adds strong requirements on feature set, performance, and durability.

\subsection{Latency}

Decentralized, distributed storage has massive opportunities for parallelism
with transfer rates, processing, and number of other factors. Parallelism by
itself is a great way to increase overall throughput even when individual
network links are slow.

Even though parallelism can improve {\em throughput} performance, parallelism
cannot by itself improve {\em latency}. If an individual network link has fixed
latency and is a required part of an operation, the overall operation will
have at least the latency of that individual link.

A distributed system interested in high performance must aggressively and
ruthlessly optimize for low latency, both at the individual process scale and
at the overall architecture scale.

Besides attempting to reduce latency in the small scale directly in as many
components as possible, one overall architecture strategy emphasized in this
paper is to optimize latency by focusing on eliminating the need to wait for
long tails.\cite{tail-at-scale} In other words, no system request should require
waiting on the slowest peers out of a set. Every request should be able to be
satisfied by the fastest nodes out of the set participating, without waiting
for a slow subset.

Focusing on operations where the result is only dependent on the fastest nodes
turns what could be a potential liability (highly variable performance from
individual actors) into a great source of strength for a distributed storage
network.

\subsection{Device failure and churn}

For any storage system, but especially a distributed storage system, component
failure is a guarantee. All hard drives fail
after enough wear\cite{backblaze-hd-2018-q1}, and the servers providing
the network access to these hard drives will eventually fail, too. Network
links die, power is lost, and storage mediums become unreliable. For data
to outlast individual component failure, data must be stored with enough
redundancy to recover from failure, and perhaps most importantly, no data is
stationary and all data must eventually be moved.

In such an environment, redundancy, data maintanance, repair, and replacement
of lost redundancy are facts of life, and the system must account for these
issues.

Decentralized systems are additionally susceptible to high churn rates,
where potential participants join the network and then leave for various
reasons well before their hardware has actually failed. Despite the issues with
failure, in decentralized systems, Maymounkov et al. found that
the probability of a node staying a member of the network for an additional
hour only {\em increases} with uptime.\cite{kademlia} In other words, the
longer a node is a participant in the network, the more likely it is to
continue to participate.

As a result, a highly durable storage system must strive to keep the churn rate
as low as possible. Too high of a churn rate and the system will simply be
moving data around constantly, using up all available bandwidth. See Appendix
\todo{} for a discussion about how repair bandwidth varies as a function of
node churn and uptime.

\subsection{Bandwidth}

Even though bandwidth availability globally is increasing year over year, access
to high bandwidth internet connections is unevenly distributed, and while some
countries have good access to symmetric, high-speed, unlimited bandwidth, other
countries have significant widespread problems.

In the United States, many residential internet service providers provide
internet with two specific problems. The first is that the internet connection
is often asymmetric, where customers buy internet based on the advertised
download speed and the upload speed is potentially an order of magnitude or
two slower. The second is that the bandwidth is sometimes "capped" at a fixed
amount of traffic per month. An internet connection that gets 10 megabytes per
second of throughput with a cap of one terabyte per month is actually only able
to use 414 kilobytes on a consistent basis all month without going over the
bandwidth cap. Bandwidth caps impose a significant limit to the amount of
available bandwidth throughout the month.

With all of this guaranteed repair traffic, it's important to make sure there
is enough headroom for the bandwidth required by data maintenance, over and
above the bandwidth required for data storage and retrieval.

To design a storage system that is careless with bandwidth usage would be to
abdicate that system to just the hands of the storage providers with access to
unlimited, high-speed bandwidth, thus recentralizing the system to some degree.
To keep the storage system as decentralized as possible and to work in as many
environments as possible, bandwidth usage must be aggressively minimized.

Please see Appendix \todo{} for a discussion on how available bandwidth,
combined with required repair traffic, limits usable space.

\section{Design}

Storj is a protocol that creates a distributed network for the reliable storage
of data and facilitates payment for successful data storage between peers. The
Storj protocol enables peers on the network to transfer data, verify the
integrity and availability of remote data, retrieve data, and pay other nodes
for storing data.
Each peer is an autonomous agent, capable of performing these actions without
significant human interaction.

At a high level, there are three major operations in the system: storing,
retrieving, and maintaining data.

\begin{description}

\item[Storing]
When data is stored with the network, the client encrypts it, breaks it up into
multiple little pieces, distributes the pieces to peers in the network, and
generates and stores some metadata about where to find the data again.

\item[Retrieving]
When data is retrieved from the network, the metadata about where to find the
pieces is recovered first, then the pieces are retrieved and the original data
is reconstructed.

\item[Maintaining]
Data is maintained in the network by replacing missing pieces when the amount
of redundancy drops below a certain threshold. The data is reconstructed and
then the missing pieces are regenerated and replaced.

\end{description}

To make this system feasible while satisfying our design constraints, we will
need to solve a number of complex challenges. Inspired by Raft\cite{raft}, we
will break the design up into a collection of relatively independent concerns
and then bring them together.

Importantly, one large benefit of breaking up
the system into this collection of concerns is it will be much easier to
vastly improve individual components without rearchitecting the rest of the
network. In the Future Work section \todo{}, we'll discuss a number of
improvements each component might adopt in the near future, but for now, our
goal is concrete proposals with clear implementation strategies for each
required component.

\begin{itemize}
\item Farmers
\item Peer-to-peer communication
\item Overlay network
\item Erasure encoding
\item Encryption
\item Structured file storage
\item Network state
\item Authorization
\item Farmer Reputation
\item Payments
\item Payer Reputation
\item Repair
\end{itemize}

\subsection{Farmers}

\todo{}

\begin{itemize}
\item Identity, proof of work (S/Kad), use ERC20 addresses
\item Get
\item Put
\item Delete
\end{itemize}

\subsection{Peer-to-peer communication}

\todo{}

\begin{itemize}
\item Signed messages (who is talking)
\item NAT traversal
\item gRPC
\end{itemize}

\todo{
Due to the presence of NATs and other adverse network conditions, not all
devices are publicly accessible. To enable non-public nodes to participate in
the network, Storj implements a reverse tunnel system.

To facilitate this system, Storj extends Kademlia with three additional message
types: PROBE, FIND\_TUNNEL, and OPEN\_TUNNEL The tunneling system also makes use
of the publish/subscribe system detailed in section 2.6.

PROBE messages allow a node to determine whether it is publically addressable.
The message is sent to a publicly addressable node, typically a known network
seed. The receiving node issues a separate PING message. The receiving node then
responds to the PROBE message with the result of the PING. Nodes joining the
network should immediately send a PROBE to any known node.

Nodes that receive a negative response to their initial PROBE should issue a
FIND\_TUNNEL request to any known node. That node must respond with three
contacts that have previously published a tunnel announcement via the
publish/subscribe system. Tunnel providers must be publicly addressable.

Once the non-public node has received a list of tunnel providers, it issues
OPEN\_TUNNEL requests to the tunnel providers. The providers must provide a
tunnel for that node if they are capable. To open a connection, the provider
sends back an affirmative response with tunnel information. The tunneled node
then opens a long-lived connection to the provider, and updates its own contact
information to reflect the tunnel address.

Tunnels are operated over TCP sockets by a custom reverse-tunneling library,
Diglet \cite{14}. Diglet provides a simple and flexible interface for
general-purpose reverse tunneling. It is accessible both by command-line and
programmatically.
}

\subsection{Overlay network}

Storj is built on Kademlia\cite{kademlia}, a distributed hash table (DHT).
We have additionally implemented the S/Kademlia\cite{s-kademlia} extensions to
mitigate a number of attack vectors. Kademlia and S/Kademlia typically provide
key/value storage in addition to peer address lookup; however, we are
only using Kademlia for peer address lookup.

\todo{}

\subsection{Erasure encoding}

\todo{}

\todo{
Cloud object stores typically own or lease servers to store their customers’
files. They use RAID schemes or a multi-datacenter approach to protect the file
from physical or network failure. Because Storj objects exist in a distributed
network of untrusted peers, farmers should not be relied upon to employ the same
safety measures against data loss as a traditional cloud storage company.
Indeed, farmers may simply turn off their node at any time. As such, it is
strongly recommended that the data owner implement redundancy schemes to ensure
the safety of their file. Because the protocol deals only with contracts for
individual shards, many redundancy schemes may be used. Three are described
below.

Storj will soon implement client-side Reed-Solomon erasure coding \cite{27}.
Erasure coding algorithms break a file into $ k $ shards, and programmatically
create $ m $ parity shards, giving a total of $ k + m = n $ shards. Any $ k $ of
these $ n $ shards can be used to rebuild the file or any missing shards.
Availability of the file is then $ P = 1 - \prod_{0}^{m} a_{m} $ across the set
of the $m + 1$ least available nodes. In the case of loss of individual shards,
the file can be retrieved, the missing shard rebuilt, and then a new contract
negotiated for the missing shard.

To prevent loss of the file, data owners should set shard loss tolerance levels.
Consider a 20-of-40 erasure coding scheme. A data owner might tolerate the loss
of 5 shards out of 40, knowing that the chance of 16 more becoming inaccessible
in the near future is low. However, at some point the probabilistic availability
will fall below safety thresholds. At that point the data owner must initiate a
retrieve and rebuild process.

Because node uptimes are known via the audit process, tolerance levels may be
optimized based on the characteristics of the nodes involved. Many strategies
may be implemented to handle this process.

Erasure coding is desirable because it drastically decreases the probability of
losing access to a file. It also decreases the on-disk overhead required to
achieve a given level of availability for a file. Rather than being limited by
the least available shard, erasure coding schemes are limited by the
least-available $ n + 1 $ nodes (see Section 6.1).
}

\subsection{Encryption}

\todo{}

\subsection{Structured file storage}

\todo{}

\begin{description}
\item[Bucket] A \x{bucket} is an unbounded but named collection of
\x{file}s identified by \x{path}s. Each \x{path} represents one
\x{file}, and every \x{file} has a unique \x{path}.

\item[Path] A \x{path} is a unique identifier for a \x{file} within a
\x{bucket}. A \x{path} is a string of UTF8 codepoints that begins with a forward
slash and ends with something besides a forward slash. More than one forward
slash (referred to as the \x{path separator}) separate \x{path components}.

An example path might be \code{/etc/hosts}, where the \x{path components} are
\code{etc} and \code{hosts}.

Clients encrypt \x{paths} before they ever leave the client computer.

\item[File] A \x{file} is a collection of \x{stream}s. Every \x{file} has
exactly one default \x{stream} and may have 0 or more named \x{stream}s.
Multiple \x{stream}s allow flexible support of extended attributes, alternate
data streams, resource forks, and other slightly more esoteric filesystem
features.

Like \x{path}s, the data contained in a \x{file} is encrypted before it ever
leaves the client computer.

\item[Stream] A \x{stream} is an ordered collection of 0 or more \x{segment}s.
\x{segment}s have a fixed maximum size, and so the more bytes the \x{stream}
represents through \x{segment}s, the more \x{segment}s there are.

\item[Segment] A \x{segment} represents a single array of bytes, between 0 and a
user-configurable maximum \x{segment} size. Breaking large \x{file}s into
multiple \x{segment}s provides a number of security and scalability advantages.

\item[Inline Segment] An \x{inline segment} is a \x{segment} that is small
enough it makes sense to store it "inline" with the metadata that keeps track of
it.

\item[Remote Segment] A \x{remote segment} is a larger \x{segment} that will be
encoded and distributed across the network. A \x{remote segment} is larger than
the metadata required to keep track of its book keeping.

\item[Stripe] A \x{stripe} is a further subdivision of a \x{segment}. A \x{stripe}
is a fixed amount of bytes that is used as an encryption and erasure encoding
boundary size. Encryption and erasure encoding happen on \x{stripe}s
individually. Encryption happens on all \x{segment}s, but erasure encoding only
happens on \x{remote segment}s.

\item[Erasure Share] When a \x{segment} is a \x{remote segment}, its \x{stripe}s
will get erasure encoded. When a \x{stripe} is erasure encoded, it generates
multiple pieces called \x{erasure share}s. Only a subset of the
\x{erasure share}s are needed to recover the original \x{stripe}, but each
\x{erasure share} has an index identifying which \x{erasure share} it is (e.g.,
the first, the second, etc.).

\item[Piece] When a \x{remote segment}'s \x{stripe}s are erasure encoded into
\x{erasure share}s, the \x{erasure share}s for that \x{remote segment} with the
same index are concatenated together, and that concatenated group of
\x{erasure share}s is called a \x{piece}. If there are $n$ \x{erasure share}s
after erasure encoding a \x{stripe}, there are $n$ \x{piece}s after processing
a \x{remote segment}. The $i$th \x{piece} is the concatenation of all of the
$i$th \x{erasure shares} from that \x{segment}'s \x{stripe}s.

\item[Piece Storage Node] A node in the network that is responsible for storing
\x{piece}s. These are operated by \x{farmer}s.

\item[Farmer] A person or group that is responsible for running and maintaining
\x{piece storage nodes}.

\item[Pointer] A \x{pointer} is a data structure that keeps track of which
\x{piece storage nodes} a \x{remote segment} was stored on, or the
\x{inline segment} data directly if applicable.

\end{description}

\subsubsection{Files as Streams}

Many applications benefit from being able to keep metadata alongside files.
For example, NTFS supports "alternate data streams" for each file, HFS supports
resource forks, EXT4 supports "extended attributes," and more importantly for
our purposes, AWS S3 supports "object metadata."\cite{s3-object-meta} Being
able to support arbitrarily named sets of keys/values dramatically improves
compatibility with other storage platforms.

Every \x{file} will have at least one \x{stream} (the default \x{stream}) and
many files may never have another \x{stream}.

\subsubsection{Streams as Segments}

Because \x{stream}s are used for data (the default \x{stream}) and metadata
(extended attributes, etc.), \x{stream}s should be designed both for small data
and large data. If a \x{stream} only has very little data, it will have one
small \x{segment}. If that \x{segment} is smaller than the metadata it would
require to store across the network, the \x{segment} will be an
\x{inline segment} and the data will be stored directly inline with the
metadata.

For larger \x{stream}s, past a certain size the data will be broken into
multiple large \x{remote segment}s. Segmenting in this manner has a number of
advantages to security, privacy, performance, and availability.

Maximum \x{segment} size is a configurable parameter. To preserve privacy, it is
recommended that \x{segment} sizes be standardized as a byte multiple, such as 8
or 32 MB. Smaller \x{segment}s may be padded with zeroes or random data.
Standardized sizes help frustrate attempts to determine the content of a given
\x{segment} and can help obscure the flow of data through the network.

Segmenting large files like video content and distributing the \x{segment}s
across the network separately reduces the impact of content delivery on any
given node.
Bandwidth demands are distributed more evenly across the network. In addition,
the end-user can take advantage of parallel transfer, similar to
BitTorrent\cite{24} or other peer-to-peer networks.

\subsubsection{Segments as Stripes}

In many situations it's important to be able to access just a portion of some
data. Some large file formats such as large video files, disk images, or file
archives support the concept of seeking, where only a partial subset of the
data is needed for correct operation. In these cases it's useful to be able to
decode and decrypt only parts of a file.

A \x{stripe} is no more than a couple of kilobytes, and encrypting and encoding
a single \x{stripe} at a time allows us to read portions of a large \x{segment}
without retrieving the entire \x{segment}, allows us to stream data into the
network without staging it beforehand, and enables a number of other useful
features.

\x{stripe}s should be encrypted client-side before being erasure encoded. The
reference implementation uses AES256-GCM by default but XSalsa20+Poly1305 is
also provided. This protects the content of the data from the \x{farmer} housing
the data. The data owner retains complete control over the encryption key, and
thus over access to the data.

It's important to use authenticated encryption to defend against data
corruption (willful or negligent), and with a monotonically increasing
nonce to defeat reordering attacks. The nonce should be monotonically increasing
across \x{segment}s throughout the \x{stream}. If \x{stripe} $i$ is encrypted
with nonce $j$, \x{stripe} $i+1$ should be encrypted with nonce $j+1$. Each
\x{segment} should get a new encryption key whenever the content in the
\x{segment} changes to avoid nonce reuse.

\subsubsection{Stripes as Erasure Shares}

Erasure encoding gives us the chance to control network durability in the face
of unreliable \x{piece storage node}s. Erasure encoding schemes often are
described as $(n, k)$ schemes, where $k$ \x{erasure shares} are needed for
reconstruction out of $n$ total. For every \x{stripe}, $n$ \x{erasure share}s
are generated, where the network has an expansion factor of $\frac{n}{k}$.

For example, let's say a \x{stripe} is broken into 40 \x{erasure share}s
($n=40$), where any 20 ($k=20$) are needed to reconstruct the \x{stripe}. Each
of the 40 \x{erasure share}s will be $\frac{1}{20}$th the size of the original
\x{stripe}.

All $n$ \x{erasure share}s have a well defined index associated with them. The
$i$th share will always be the same, given the same input parameters.

Because peers generally rely on separate hardware and infrastructure, data
failure is not correlated. This implies that erasure codes are an extremely
effective method of securing availability. Availability is proportional to the
number of nodes storing the data.

See section \todo{} for a breakdown of how varying the erasure code parameters
affects availability and redundancy.

\subsubsection{Erasure Shares as Pieces}

Because \x{stripe}s are already small, \x{erasure share}s are often
much smaller, and the metadata to keep track of all of them separately would be
immense relative to their size. Instead of keeping track of all of the shares
separately, we pack all of the \x{erasure share}s together into a few
\x{piece}s. In an $(n, k)$ scheme, there are $n$ \x{piece}s, where each
\x{piece} $i$ is the ordered concatenation of all of the \x{erasure share}s
with index $i$.

As a result, where each \x{erasure share} is an $\frac{n}{k}$th of a \x{stripe},
each \x{piece} is an $\frac{n}{k}$th of a \x{segment}, and only $k$ \x{piece}s
are needed to recover the full \x{segment}.

\subsubsection{Pointers}

The data owner will need knowledge of how a \x{remote segment} is broken up and
where in the network the \x{piece}s are located to recover it. This is contained
in the \x{pointer} data structure, and the owner can secure the \x{pointer} as
they wish. As the set of \x{segment}s in the network grows, it becomes
exponentially more difficult to locate any given \x{piece} set without prior
knowledge of their locations (see Section 6.3). This implies that security of
the \x{remote segment} is proportional to the square of the size of the network.

\subsection{Network state}

\todo{}

\subsection{Authorization}

\todo{}

\subsection{Farmer Reputation}

\todo{}

\subsection{Payments}

\todo{}

\subsection{Payer Reputation}

\todo{}

\subsection{Repair and Maintenance}

\todo{}

\subsection{Proofs of Retrievability}

\todo{
Proofs of retrievability guarantee the existence of a certain piece of data on a
remote host. The ideal proof minimizes message size, can be calculated quickly,
requires minimal pre-processing, and provides a high degree of confidence that
the file is available and intact. To provide knowledge of data integrity and
availability to the data owner, Storj provides a standard format for issuing and
verifying proofs of retrievability via a challenge-response interaction called
an “audit” or “heartbeat.”

Our reference implementation uses Merkle trees \cite{5} and Merkle proofs. After
the sharding process the data owner generates a set of $ n $ random challenge
salts $ s_{0}, s_{1}, ... s_{n-1} $ and stores the set of salts $ s $. The
challenge salts are each prepended to the data $ d $, and the resulting string
is hashed to form a “pre-leaf” $ p $ as such: $ p_{i} = H(s_{i} + d) $. Salts
are prepended, rather than appended, in order to defeat length extension
attacks. Pre-leaves are hashed again, and the resulting digests become the set
of leaves $ l $ of a standard Merkle tree such that $ l_{i} = H(H(s_{i} + d)) $.
The leaf set is filled with hashes of a blank string until its cardinality is a
power of two, to simplify the proof process.

%insert Figure 2
\begin{figure}[hbt]
\centering
\includegraphics[width=\linewidth]{2}
\caption{Storj Audit Tree with $ |l| = 4 $\\Red outlines indicate the elements
of a Merkle proof for $ s_{0} $}
\end{figure}

The data owner stores the set of challenges, the Merkle root and the depth of
the Merkle tree, then transmits the Merkle tree’s leaves to the farmer. The
farmer stores the leaves along with the shard. Periodically, the data owner
selects a challenge from the stored set, and transmits it to the farmer.
Challenges may be selected according to any reasonable pattern, but should not
be reused. The farmer uses the challenge and the data to generate the pre-leaf.
The pre-leaf, along with the set of leaves, is used to generate a Merkle proof,
which is sent back to the data owner.

The Storj Merkle proof always consists of exactly $ log_{2}(|l|)+1 $ hashes, and
thus is a compact transmission, even for large trees. The data owner uses the
stored Merkle root and tree depth to verify the proof by verifying that its
length is equal to the tree depth and the hashes provided recreate the stored
root. This scheme does not allow false negatives or false positives, as the hash
function requires each bit to remain intact to produce the same output.
}

\subsubsection{Partial Audits}

\todo{
The Merkle tree audit scheme requires significant computational overhead for the
data owner, as the entire shard must be hashed many times to generate
pre-leaves. An extension of this scheme utilizes subsets of the data to perform
partial audits, reducing computational overhead. This also has the advantage of
significantly reducing I/O burden on farmer resources.

This extension relies on two additional selectable parameters: a set of byte
indices $ x $ within the shard and a set of section lengths in bytes, $ b $. The
data owner stores a set of 3-tuples $ (s, x, b) $. To generate pre-leaf $ i $,
the data owner prepends $ s_{i} $ to the $ b_{i} $ bytes found at $ x_{i} $.
During the audit process, the verifier transmits $ (s, x, b)_{i} $, which the
farmer uses to generate a pre-leaf. The Merkle proof is generated and verified
as normal.

%insert Figure 3
\begin{figure}[hbt]
\centering
\includegraphics[width=\linewidth]{3}
\caption{Storj Audit Tree with $ |l| = 4 $ and Partial Audits\\Red outlines
indicate the elements of a Merkle proof for $ s_{0} $}
\end{figure}

Partial audits provide only probabilistic assurance that the farmer retains the
entire file. They allow for false positive results, where the verifier believes
the farmer retains the intact shard, when it has actually been modified or
partially deleted. The probability of a false positive on an individual partial
audit is easily calculable (see Section 6.4)

Thus the data owner can have a known confidence level that a shard is still
intact and available. In practice, this is more complex, as farmers may
implement intelligent strategies to attempt to defeat partial audits.
Fortunately, this is a bounded problem in the case of iterative audits. The
probability of several consecutive false positives becomes very low, even when
small portions of the file have been deleted.

In addition, partial audits can be easily mixed with full audits without
restructuring the Merkle tree or modifying the proof verification process. Many
audit strategies that mix full and partial verification can be envisioned, each
of which provides different levels of confidence over time.

A further extension of this scheme could use a deterministic seed instead of a
set of byte indexes. This seed would be used to generate indexes of many
non-consecutive bytes in the file. Requiring many non-consecutive random bytes
would provide additional resistance against malicious farmers attempting to
implement audit evasion strategies without significant extra overhead from
processing or I/O.
}

\subsubsection{Other Proof-of-Retrievability Schemes}

\todo{
Other audit schemes were examined, but deemed generally infeasible. For example,
Shacham and Waters proposed a compact proof \cite{6} with several advantages
over Merkle-tree schemes. This construction allows for an endless stream of
challenges to be generated by the data owner with minimal stored information. It
also allows for public verifiability of challenge responses.

However, initial implementations indicate that the client-side pre-processing
required for the Shacham-Waters scheme requires at least one order of magnitude
more computation time than hash-based methods, rendering it too slow for most
applications.

Proof of retrievability is an area of ongoing research, and other practical
schemes may be discovered in the future. As proof of retrievability schemes are
discovered and implemented, the choice of scheme may become a negotiable
contract parameter. This would allow each data owner and node to implement a
wide variety of schemes, and select the most advantageous scheme for a given
purpose.
}

\subsubsection{Issuing Audits}

\todo{
To issue audits, Storj extends the Kademlia message set with a new type: AUDIT
(for a full list of Kademlia extensions, see Appendix A). These messages are
sent from data owners to farmers and contain the hash of the data and a
challenge. The farmer must respond with a Merkle proof as described above. Upon
receipt and validation of the Merkle proof, the data owner must issue payment to
the farmer according to agreed-upon terms.

%insert Figure 4
\begin{figure}[hbt]
\centering
\includegraphics[width=\linewidth]{4}
\caption{Issuing and Verifying Storj Audits}
\end{figure}
}

\subsection{Contracts and Negotiation}

\todo{
Data storage is negotiated via a standard contract format \cite{7}. The contract
is a versioned data structure that describes the relationship between data owner
and farmer. Contracts should contain all information necessary for each node to
form a relationship, transfer the data, create and respond to audits over time,
and arbitrate payments. This includes shard hash, shard size, audit strategy,
and payment information. Storj implements a publish/subscribe system to connect
parties interested in forming a contract (see Section 2.6).

Each party should store a signed copy of the contract. Contracts exist solely
for the benefit of the data owner and farmer, as no other node can verify the
terms or state of the relationship. In the future, contract information may be
stored in the DHT, or in an external ledger like a Blockchain, which may allow
some outside verification of relationship terms.

The contracting system extends Kademlia with four new message types: OFFER,
CONSIGN, MIRROR, and RETRIEVE.

To negotiate a contract, a node creates an OFFER message and sends it to a
prospective partner. Prospective partners are found via the publish/subscribe
system described in Section 2.6. The OFFER message contains a fully-constructed
contract that describes the desired relationship. The two nodes repeatedly swap
signed OFFER messages. For each new message in the OFFER loop, the node either
chooses to terminate negotiations, respond with a new signed counter-offer, or
accept the contract by countersigning it. Once an OFFER is signed by both
parties, they each store it locally, keyed by the hash of the data.

Once an agreement is reached, the data owner sends a CONSIGN message to the
farmer. The message contains the leaves of the audit-tree. The farmer must
respond with a PUSH token that authorizes the data owner to upload the data via
HTTP transfer (see Section 2.10). This token is a random number, and can be
delegated to a third party.

RETRIEVE messages signify the intent to retrieve a shard from a farmer. These
messages are nearly identical to CONSIGN messages, but do not contain the
audit-tree leaves. The farmer responds to a valid RETRIEVE message with a PULL
token that authorizes download of the data via a separate HTTP transfer.

MIRROR messages instruct a farmer to retrieve data from another farmer. This
allows data owners to create redundant copies of a shard without expending
significant additional bandwidth or time. After a successful OFFER/CONSIGN
process, the data owner may initiate a separate OFFER loop for the same data
with another farmer. Instead of issuing a CONSIGN message to the mirroring
farmer, the data owner instead issues a RETRIEVE message to the original farmer,
and then includes the retrieval token in a MIRROR message to the mirroring
farmer. This authorizes the mirroring farmer to retrieve the data from the
original farmer. The success of the MIRROR process should be verified
immediately via an AUDIT message.
}

\subsection{Payment}

\todo{
Storj is payment agnostic. Neither the protocol nor the contract requires a
specific payment system. The current implementation assumes Storjcoin, but many
other payment types could be implemented, including BTC, Ether, ACH transfer, or
physical transfer of live goats.

The reference implementation will use Storjcoin micropayment channels, which are
currently under development \cite{26}. Micropayment channels allow for pairing
of payment directly to audit, thus minimizing the amount of trust necessary
between farmers and data owners. However, because data storage is inexpensive,
audit payments are incredibly small, often below \$0.000001 per audit.

Storjcoin allows much more granular payments than other candidate currencies,
thereby minimizing trust between parties. In addition, the mechanics of
micropayment channels require the total value of the channel to be escrowed for
the life of the channel. This decreases currency velocity, and implies that
value fluctuations severely impact the economic incentives of micropayment
channels. The use of a separate token creates a certain amount of insulation
from outside volatility, and Storjcoin's large supply minimizes the impact of
token escrow on the market.

New payment strategies must include a currency, a price for the storage, a price
for retrieval, and a payment destination. It is strongly advised that new
payment strategies consider how data owners prove payment, and farmers verify
receipt without human interaction. Micropayment networks, like the Lightning
Network \cite{25}, solve many of these problems, and are thus an ideal candidate
for future payment strategies. Implementation details of other payment
strategies are left as an exercise for interested parties.
}

\section{Services}

\todo{
As should be apparent, the data owner has to shoulder significant burdens to
maintain availability and integrity of data on the Storj network. Because nodes
cannot be trusted, and hidden information like challenge sets cannot be safely
outsourced to an untrusted peer, data owners are responsible for negotiating
contracts, pre-processing shards, issuing and verifying audits, providing
payments, managing file state via the collection of shards, managing file
encryption keys, etc. Many of these functions require high uptime and
significant infrastructure, especially for an active set of files. User run
applications, like a file syncing application, cannot be expected to efficiently
manage files on the network.

To enable simple access to the network from the widest possible array of client
applications, Storj implements a thin-client model that delegates trust to a
dedicated server that manages data ownership. This is similar to the SPV wallet
concept found in Bitcoin and other cryptocurrency ecosystems. The burdens of the
data owner can be split across the client and the server in a variety of ways.
By varying the amount of trust delegated, the server could also provide a wide
variety of other valuable services. This sort of dedicated server, called
Bridge, has been developed and released as Free Software. Any individual or
organization can run their own Bridge server to facilitate network access.
}

\subsection{Farmer}

\todo{}

\subsection{Bridge}

\todo{
Our reference implementation of this model consists of a Bridge server, and a
client library. Bridge provides an object store, which is to say, the primary
function of Bridge is to expose an API to application developers. Developers
should be able to use the Bridge via a simple client without requiring knowledge
of the network, audit procedures, or cryptocurrencies. The Bridge API is an
abstraction layer that streamlines the development process. This enables
developers to create many applications that use the Storj network, allowing the
network to reach many users.

In the current implementation, Bridge assumes responsibility for contract
negotiation, audit issuance and verification, payments, and file state, while
the client is responsible for encryption, pre-processing, and file key
management. The Bridge exposes access to these services through a RESTful API.
In this way, the client can be completely naive of the Storj protocol and
network while still taking advantage of the network. In addition, because the
dedicated server can be relied on to have high uptime, the client can be
integrated into unreliable user-space applications.

Bridge is designed to store only metadata. It does not cache encrypted shards
and, with the exception of public buckets, does not hold encryption keys. The
only knowledge of the file that Bridge is able to share with third parties is
metadata such as access patterns. This system protects the client's privacy and
gives the client complete control over access to the data, while delegating the
responsibility of keeping files available on the network to Bridge.

It is possible to envision Bridge upgrades that allow for different levels of
delegated trust. A Bridge client may want to retain control over issuing and
validating audits, or managing pointers to shards. Or a client may choose to
authorize two or more unrelated Bridges to manage its audits in order to
minimize the trust it places in either Bridge server. In the long run, any
function of the data owner can be split across two or more parties by delegating
trust.
}

\subsection{Client Library}

\todo{
Full documentation of the Bridge API is outside the scope of this whitepaper,
but is available elsewhere \cite{16}. The first complete client implementation
is in JavaScript. Implementations in C, Python, and Java are in progress.

Because files cannot simply be POSTed to API endpoints, the structures of the
Bridge API and client are different from existing object stores. Clients are
implemented to hide the complexity of managing files on the storj network
through simple and familiar interfaces. As much as possible, complex network
operations are abstracted away behind simple function calls.

A brief summary of the upload process follows:

\begin{enumerate}
\item The client gathers and pre-processes data.
\item The client notifies Bridge of data awaiting upload.
\item Bridge negotiates contracts with network nodes.
\item Bridge returns the IP addresses of contracted nodes, and authorization
tokens to the client.
\item The client uses the IP addresses and tokens to contact the farming nodes
and upload the data.
\item The client transfers the audit information to the Bridge, delegating
trust.
\item Bridge immediately issues an audit and verifies the response, to prove
data was transferred correctly.
\item Bridge assumes responsibility for issuing audits, paying farmers, and
managing file state.
\item Bridge exposes file metadata to the client via the API.
\end{enumerate}

The download process is similar.

\begin{enumerate}
\item The client requests a file by an identifier.
\item Bridge validates the request and provides a list of farmer IP addresses
and tokens.
\item The client uses the addresses and tokens to retrieve the file
\item The file is reassembled and decrypted client-side.
\end{enumerate}

The JavaScript library accepts file, handles pre-processing, and manages
connections as directed by Bridge. It also makes decrypted downloads available
to applications as files, or as streams. A sample CLI using the library is
available as Free Software at https://github.com/storj/core-cli. It has been
tested with a wide variety of file sizes, and is capable of reliably streaming
1080p video from the Storj network.
}

\subsubsection{Application Development Tools}

\todo{
The primary function of Bridge and the Bridge API is to serve applications. To
this end clients and tools in a wide variety of languages are under development.

Storj.js\cite{28} seeks to provide a standard in-browser interface for
downloading files from Storj. Though in early stages, it can already communicate
with Bridge, retrieve file pointers and tokens, retrieve shards from farmers,
reassemble shards, and append the completed file to the DOM. This allows web
developers to easily reference Storj objects from within a page, and rely on
them being delivered properly to the end user. This could be used to provide any
service from in-browser document editing to photo storage.

Key and file management tools for web backends are in early planning stages,
including Storj plugins for standard backend tools like content management
systems. These tools should help content-driven application developers work with
files on the Storj network. Standardizing these tools around permissioning files
by user could help create data portability between services as discussed in
section 4.2.

Bridges to other protocols and workflows are also planned. The Storj CLI lends
itself to shell scripting automation. Similar tools for FTP, FUSE, and common
tools for interacting with files will be developed in the future.
}

\subsection{Bridge as an Authorization Mechanism}

\todo{
Bridge can be used to manage authorization for private files stored on the
network. Because Bridge manages the state of each contract under its care, it is
a logical provider of these services. It can manage a variety of
authorization-related services to enable sharing and collaboration.
}

\subsubsection{Identity and Permissioning}

\todo{
The Bridge API uses public-key cryptography to verify clients. Rather than the
Bridge server issuing an API key to each user, users register public keys with
the Bridge. API requests are signed, and the Bridge verifies that the signature
matches a registered public key. Bridge organizes file metadata into buckets to
facilitate management. Buckets can be permissioned individually by registering a
set of public keys to the Bucket.

Application developers can use this to easily delegate permissions to
applications, servers, or other developers. For instance, the developer of a
file syncing service could create a keypair for each user of that service, and
divide each user into a separate Bucket accessible only by that user’s keypair.
Usage of each Bucket is tracked separately, so users who have exceeded their
allotment could have write permissions revoked programmatically. This provides a
logical separation of user permissions, as well as a variety of organizational
tools.
}

\subsubsection{Key Migration}

\todo{
Because shard encryption keys are stored on the device that generated them, data
portability is an issue. The reference implementation of Bridge and the client
facilitate the transfer of file encryption keys between clients in a safe way.
Clients generate a cryptographically strong seed, by default a randomly
generated twelve word phrase. To encrypt a given file, the client generates a
key deterministically based on the seed, Bucket ID and File ID.

The user can import the seed one time to each new device, which permanantly
keeps the devices syncronized. This also facilitates backup since users only
have to store the seed, not every newly generated file key.
}

\subsubsection{Public Files}

\todo{
Bridge, like other object stores, allows developers to create and disseminate
public files via public Buckets. The Bridge server allows the developer to
upload the encryption key, and then allows anonymous users to retrieve the file
key and the set of file pointers. Public Buckets are useful for content delivery
to webpages, or to public-facing applications.

A system to share and retrieve public files without need of a Bridge could also
be created. Pointers and keys could be posted publicly on any platform, and
clients could be required to pay farmers directly for downloads. In practice
this would be very similar to an incentivized torrent. Platforms serving
pointers function similarly to trackers facilitating torrents. It is unclear
whether this system would have significant advantages over existing torrent
networks.
}

\subsubsection{File Sharing}

\todo{
In the future, the Bridge could enable sharing of specific files between
applications or users. Because all files co-exist on a shared network, this is a
problem of standardization and identity management.

Bridge could also use a third-party source of identity, like a PGP keyserver or
Keybase\cite{29}, to enable secure person-to-person file sharing. A tiered
keying strategy (as used by LastPass\cite{28}) could also allow for the sharing
of individual files. Other cryptographic schemes like proxy re-encryption seem
promising. For a simplified example: if file keys are strongly encrypted and
escrowed with a Bridge, files could be shared to any social media handle that
could be authenticated via Keybase. Bridge could send the corresponding client a
single encrypted file key along with a transposition key, thus enabling access
to a file without exposing the file to Bridge, or modifying the file in any way.

A thorough description of these key management schemes is outside the scope of
this paper. It is enough to note that they exist, that many useful strategies
can be implemented in parallel, and that a dedicated Bridge can facilitate them
in many useful ways.
}

\subsection{Bridge as a Network Information Repository}

\todo{
As noted earlier, data owners are responsible for negotiating contracts and
managing file state. With enough information about peers on the network,
contract selection becomes a powerful tool for maintaining file state. A Bridge
will have many active contracts with many farmers, and will therefore have
access to information about those farmers. A Bridge could use this information
to intelligently distribute shards across a set of farmers in order to achieve
specific performance goals.

For instance, via the execution of a contract, a Bridge node gathers data about
the farmer’s communication latency, audit success rate, audit response latency,
and availability. With minimal additional effort, the Bridge could also gather
information about the node’s available bandwidth. By gathering a large pool of
reliable data about farmers, a Bridge node can intelligently select a set of
farmers that collectively provides a probabilistic guarantee of a certain
quality of service.

In other words, the Bridge can leverage its knowledge about peers on the network
to tailor the service to the client’s requirements. Rather than a limited set of
service tiers, a Bridge could assemble a package of contracts on the fly to meet
any service requirement. This allows the client to determine the optimal
latency, bandwidth, or location of a file, and have confidence that its goals
will be met. For instance, a streaming video application may specify a need for
high bandwidth, while archival storage needs only high availability. In a
sufficiently large network, any need could be met.

Secure distributed computation is an unsolved problem and, as such, each Bridge
server uses its accumulated knowledge of the network. The Bridge is able to
provide a probabilistic quality of service based on its knowledge the
performance and reliability of farmers that a distributed network alone cannot
provide.
}

\subsection{Bridge as a Service}

\todo{
In cases where the cost of delegating trust is not excessively high, clients may
use third-party Bridges. Because Bridges do not store data and have no access to
keys, this is still a large improvement on the traditional data-center model.
Many of the features Bridge servers provide, like permissioning and intelligent
contracting, leverage considerable network effects. Data sets grow exponentially
more useful as they increse in size, indicating that there are strong economic
incentives to share infrastructure and information in a Bridge.

Applications using object stores delegate significant amounts of trust to the
storage providers. Providers may choose to operate public Bridges as a service.
Application developers then delegate trust to the Bridge, as they would to a
traditional object store, but to a lesser degree. Future updates will allow for
various distributions of responsibilities (and thus levels of trust) between
clients and Bridges. This shifts significant operational burdens from the
application developer to the service-provider. This would also allow developers
to pay for storage with standard payment mechanisms, like credit cards, rather
than managing a cryptocurrency wallet. Storj Labs Inc. currently provides this
service.
}

\subsection{S3 gateway}

\todo{}

\section{Future Areas of Research}

\todo{
Storj is a work in progress, and many features are planned for future versions.
There are relatively few examples of functional distributed systems at scale,
and many areas of research are still open.
}

\subsection{Fast Byzantine Consensus}

\todo{}

\subsection{Distributed Repair}

\todo{}

\subsection{Federated Bridges}

\todo{
Bridge nodes could cooperate to share data about the network in a mutually
beneficial federation. This would allow each Bridge to improve the quality of
service that it provides by improving the quality of information available.

Bridges could also, with the consent of users, cooperate to share file metadata
and pointers among themselves. This would allow a user to access their file from
any Bridge, rather than being dependent on a single Bridge. A tiered set of
fallback Bridges storing the same access information is a desirable feature, as
it hedges against downtime from a solo Bridge. Some solvable permissioning
issues may exist, but there is no reason to believe a standard format and
algorithm for syncing state across Bridges may not be developed.
}

\subsection{Data Portability}

\todo{
By encouraging use of data format and access standards, Storj aims to allow
portability of data between applications. Unlike a traditional model, where
control of data is tied to the service used to access the data, data access may
be tied to individual users because Storj forms a common underlying layer. User
data can be tied to persistent cryptographic identities, and authenticated
without exposing data to third parties. Siloing data in applications is a
harmful relic of traditional models. Building cross-compatibility into the
future of data storage greatly improves user privacy and user experience.

Applications implementing these standards would be broadly compatible. When
access is tied to users rather than services, privacy and control are preserved.
A user may grant access to a service that backs up their hard drive, which
places those files in Storj. The user could separately grant access to a
photo-sharing service, which could then access any photos in the backup. The
user gains seamless portability of data across many applications, and
application developers gain access to a large pool of existing users.

Permissioning in this system may be managed by a service like a Bridge, tied to
a web of trust identity via services like Keybase, or handled by a distributed
self-sovereign identity system. Smart contract systems, e.g. Ethereum \cite{17}
contracts, seem like a sensible long-term choice, as they can provide file
permissions based on arbitrary code execution. Some problems may exist with
respect to management of the private information required for identity and
permissioning systems, but sufficient solutions likely exist.

While this system represents a significant step up in both usability and value,
there are unmitigable security issues. Unfortunately, as in any cryptographic
system, it is impossible to revoke access to data. Applications may cache data
or forward it to third parties. Users, by definition, trust application
developers to handle their data responsibly. To mitigate these risks, Storj Labs
intends to provide incentives to developers to build free and open-source
software. No application can be completely secure, but auditable code is the
best defense of user’s privacy and security.

The potential advantages in terms of user experience and privacy are great, but
more research is needed. Many open questions exist with respect to permissioning
mechanisms. At worst a unified backend powering interoperable applications
provides equivalent security to current data-center based models. Storj hopes to
collaborate with other forward-thinking data-driven projects to create and
advocate for these open standards.
}

\subsection{Reputation Systems}

\todo{
Storj, like many distributed networks, would profit immensely from a distributed
reputation system. A reliable means of determining reputation on a distributed
system is an unsolved problem. Several approaches have been detailed, and some
implemented in practice but none have achieved consensus among researchers or
engineers. A brief review of several of these approaches follows.

One inherent downside of distributing information across a network is the
additional latency required for decisionmaking. It is difficult to say whether
any distributed reputation system can accurately assess the bandwidth, latency,
or availability of peers on a distributed network in a manner suitable to object
storage, especially as market demand for these shifts over time. Nevertheless, a
reliable distributed reputation would be an extremely useful tool for
interacting with and understanding the network.
}

\subsubsection{Eigentrust and Eigentrust++}

\todo{
Eigentrust \cite{19} attempts to generalize the ledger approach to generate
global trust values in a distributed system using a transitive-trust model.
Nodes keep and exchange trust vectors. For networks with a large majority of
trustworthy peers, the value of each local trust vector converges to a shared
global trust vector as nodes learn more about the network via information
exchange.

Eigentrust++ \cite{20} identifies several attack vectors and modifies Eigentrust
to improve performance and reliability in the presence of malicious nodes.
Eigentrust++ is currently implemented in NEM \cite{21}. Secure global
convergence to a shared trust value for each node is a key feature for any
distributed reputation system.
}

\subsubsection{TrustDavis}

\todo{
TrustDavis \cite{22} implements reputation as insurance. Nodes provide
references for other nodes in the form of insurance contracts. Nodes seeking a
prospective partner for an economic transaction also seek insurance contracts
protecting them from the actions of that partner. Reputation in this system may
be thought of as a graph, with vertices representing nodes, and directed edges
representing the monetary value that a node is willing to stake on behalf of
another. Nodes that are distant in this graph may still transact by purchasing a
set of insurance contracts that traverses these edges. TrustDavis in practice
thus encounters the same routing problem found on other distributed systems like
the Lightning Network.

Denominating trust in terms of monetary value is attractive for an economic
network like Storj, but the mechanics of insurance contracts in a system like
this represent an extremely difficult problem. Notably, because failures to
deliver payment propagate backwards through the insurance route, the financial
burden always falls on the node that trusted an untrustworthy node, rather than
the untrustworthy nodes.
}

\subsubsection{Identity Maintenance Costs}

\todo{
Storj is exploring a reputation system that leverages public Blockchains to
solve a narrow set of identity problems \cite{23}. This system requires nodes to
spend money directly to maintain reputation. Nodes invest in their identity over
time by making small standardized payments to their own Storj network node ID.
Because the ID is a Bitcoin address to which the node holds the private key,
these funds are fully recoupable, except for the miners’ fees. In this system,
nodes prefer to interact with nodes that have a long history of regular
transactions. Over time these indicate monetary investment in an identity equal
to the sum of the miners’ fees paid.

The payment required to participate in this system should be significantly less
than the expected return of operating a network node. If set correctly, this
recurring monetary payment for an identity bounds the size and duration of Sybil
attacks without affecting cooperative nodes. Legitimate nodes would easily
recoup their identity expense, while Sybil operators would find their expenses
outstripping their returns. Unfortunately, this approach solves a relatively
small subset of identity issues on the network, and it is difficult to see how
it could be extended to other problem sets.
}

\subsection{OFFER Loop Strategies}

\todo{
Many negotiation strategies can exist and interact via the OFFER loop. Full
exploration of negotiation strategies is beyond the scope of this paper, but a
few interesting areas are immediately apparent. Simple examples include price
floors and ceilings, but complex models could be built to base strategies on
market trends and the subjective value of a shard. Negotiation strategies
executed by autonomous agents are an area of (fascinating) ongoing research.
Storj will be one of the first large-scale machine-driven marketplaces. As such,
improving negotiation efficiency is critical to the long-term efficiency of the
market.
}

\section{Attacks}

As with any distributed system, a variety of attack vectors exist. Many of these
are common to all distributed systems. Some are storage-specific, and will apply
to any distributed storage system.

\subsection{Spartacus}

\todo{
Spartacus attacks, or identity hijacking, are possible on Kademlia. Any node may
assume the identity of another node and receive some fraction of messages
intended for that node by simply copying its Node ID. This allows for targeted
attacks against specific nodes and data. This is addressed by implementing Node
IDs as ECDSA public key hashes and requiring messages be signed. A Spartacus
attacker in this system would be unable to generate the corresponding private
key, and thus unable to sign messages and participate in the network.
}

\subsection{Sybil}

Sybil attacks involve the creation of large amounts of nodes in an attempt to
disrupt network operation by hijacking or dropping messages. Kademlia, because
it relies on message redundancy and a concrete distance metric, is reasonably
resistant to Sybil attacks. A node’s neighbors in the network are selected by
Node ID from an evenly distributed pool, and most messages are sent to at least
three neighbors. If a Sybil attacker controls 50\% of the network, it
successfully isolates only 12.5\% of honest nodes. While reliability and
performance will degrade, the network will still be functional until a large
portion of the network consists of colluding Sybil nodes.

\subsubsection{Google}

The Google attack, or nation-state attack, is a hypothetical variant of the
Sybil attack carried out by an entity with extreme resources. Google attacks are
hard to address, as it is difficult to predict the actions of an organization
with orders of magnitude more resources than the sum of the resources of network
participants. The only reliable defence against a Google attack is to create a
network whose resources are on the same order of magnitude as the attacker’s. At
that scale, any attack against the network would represent an unsustainable
commitment of resources for such an organization.

\subsubsection{Honest Geppetto}

The Honest Geppetto attack is a storage-specific variant of the Google attack.
The attacker operates a large number of ‘puppet’ nodes on the network,
accumulating trust and contracts over time. Once he reaches a certain threshold
he pulls the strings on each puppet to execute a hostage attack with the data
involved, or simply drops each node from the network. Again, the best defence
against this attack is to create a network of sufficient scale that this attack
is ineffective. In the meantime, this can be partially addressed by relatedness
analysis of nodes. Bayesian inference across downtime, latency and other
attributes can be used to assess the likelihood that two nodes are operated by
the same organization, and data owners can and should attempt to distribute
shards across as many unrelated nodes as possible.

\subsection{Eclipse}

\todo{mention S/Kademlia

An eclipse attack attempts to isolate a node or set of node in the network
graph, by ensuring that all outbound connections reach malicious nodes. Eclipse
attacks can be hard to identify, as malicious nodes can be made to function
normally in most cases, only eclipsing certain important messages or
information. Storj addresses eclipse attacks by using public key hashes as Node
IDs. In order to eclipse any node in the network, the attacker must repeatedly
generate key pairs until it finds three keys whose hashes are closer to the
targeted node than its nearest non-malicious neighbor, and must defend that
position against any new nodes with closer IDs. This is, in essence, a
proof-of-work problem whose difficulty is proportional to the number of nodes in
the network.

It follows that the best way to defend against eclipse attacks is to increase
the number of nodes in the network. For large networks it becomes prohibitively
expensive to perform an eclipse attack (see Section 6.2). Furthermore, any node
that suspects it has been eclipsed may trivially generate a new keypair and node
ID, thus restarting the proof-of-work challenge.
}

\subsubsection{Tunnel Eclipse}

\todo{
Because tunneled connections rely on the tunnel provider, it is trivial for a
tunnel provider to eclipse nodes for which it provides tunneled connections.
This attack cannot affect publicly addressable nodes, so it can be trivially
defeated with proper configuration. This attack can be mitigated by encrypting
messages intended for tunneled nodes, thus removing the malicious tunnel
provider's ability to inspect and censor incoming messages. Like a typical
eclipse attack, any node that suspects it is the victim of a tunnel eclipse can
easily generate a new Node ID, and find a new tunnel.
}

\subsection{Hostage Bytes}

The hostage byte attack is a storage-specific attack where malicious farmers
refuse to transfer shards, or portions of shards, in order to extort additional
payments from data owners. Data owners should protect themselves against hostage
byte attacks by storing shards redundantly across several nodes (see Section
2.7). As long as the client keeps the bounds of its erasure encoding a secret,
the malicious farmer cannot know what the last byte is. Redundant storage is not
a complete solution for this attack, but addresses the vast majority of
practical applications of this attack. Defeating redundancy requires collusion
across multiple malicious nodes, which is difficult to execute in practice.

\subsection{Cheating Owner}

\todo{
A data owner may attempt to avoid paying a farmer for data storage by refusing
to verify a correct audit. In response the farmer may drop the data-owner’s
shard. This attack primarily poses a problem for any future distributed
reputation system, as it is difficult for outside observers to verify the claims
of either party. There is no known practical publicly verifiable proof of
storage, and no known scheme for independently verifying that a privately
verifiable audit was issued or answered as claimed. This indicates that a
cheating client attack is a large unsolved problem for any reputation system.
}

\subsection{Faithless Farmer}

\todo{
While the farming software is built to require authentication via signature and
token before serving download requests, it is reasonable to imagine a
modification of the farming software that will provide shards to any paying
requestor. In a network dominated by faithless farmers, any third-party can
aggregate and inspect arbitrary shards present on the network.

However, even should faithless farmers dominate the network, data privacy is not
significantly compromised. Because the location of the shards that comprise a
given file is held solely by the data owner, it is prohibitively difficult to
locate a target file without compromising the owner (see Section 6.3). Storj is
not designed to protect against compromised data owners. In addition, should a
third-party gather all shards, strong client-side encryption protects the
contents of the file from inspection. The pointers and the encryption key may be
secured separately. In the current implementation of Bridge, the pointers and
the keys are held by the Bridge and the client, respectively.
}

\subsection{Defeated Audit Attacks}

\todo{
A typical Merkle proof verification does not require the verifier to know the
depth of the tree. Instead the verifier is expected to have the data being
validated. In the Storj audit tree, if the depth is unknown to the verifier the
farmer may attack the verification process by sending a Merkle proof for any
hash in the tree. This proof still generates the Merkle root, and is thus a
valid proof of some node. But, because the verifier does not hold the data used
to generate the tree, it has no way to verify that the proof is for the specific
leaf that corresponds to the challenge. The verifier must store some information
about the bottom of the tree, such as the depth of the tree, the set of leaves
nodes, or the set of pre-leaves. Of these, the depth is most compact, and thus
preferable.

Using the pre-leaf as an intermediary defeats another attack, where the farmer
simply guesses which leaf corresponds to the current challenge. While this
attack is unlikely to succeed, it’s trivially defeated by forcing the farmer to
provide the pre-leaf. The farmer cannot know the pre-leaf before the challenge
is issued. Requiring transmission of the pre-leaf also allows the data owner to
proceed through the challenge set linearly instead of being forced to select
randomly. This is desireable because it allows the data owner to maintain less
state information per tree.
}

\section{Selected Calculations}

The following are several interesting calculations related to the operation of
the network.

\subsection{Failure of k-of-n Erasure Coding}

The chance of failure of k-of-n erasure coding, assuming probability $ p $ every
shard stays online, is calculated as a binomial distribution:

{\centering
$\Pr_{failure}(n; k,p) = \displaystyle \sum_{i=0}^{k-1} p^{i}(1-p)^{n-i}{n
\choose i}$
\\}

%code formatting
\begin{table}[hbt!]
\begin{center}
\begin{tabular}{r r l r}
n & k & p & $\Pr_{failure}{n,k,p}$\\
\hline  18 & 6  &  0.5  & 4.812e-02\\
\hline  18 & 6  &  0.75 &3.424e-05\\
\hline  18 & 6  &  0.9  & 5.266e-10\\
\hline  18 & 6  &  0.98 &6.391e-19\\
\hline  36 & 12 &  0.5  &1.440e-02\\
\hline  36 & 12 &  0.75 &2.615e-08\\
\hline  36 & 12 &  0.9  &1.977e-17\\
\hline  36 & 12 &  0.98 &1.628e-34\\
\end{tabular}
\end{center}
\end{table}

Code:
\begin{lstlisting}
def fac(n): return 1 if n==0 else n * fac(n-1)
def choose(n,k): return fac(n) / fac(k) / fac(n-k)
def bin(n,k,p): return choose(n,k) * p ** k * (1-p) ** (n-k)
def prob_fail(n,k,p): return sum([bin(n,i,p) for i in range(0,k)])
\end{lstlisting}

Therefore, with well-chosen $ k $ and $ n $, in addition to recovery methods
described above, the statistical chance of shard or file loss is quite small.

\subsection{Difficulty of Eclipsing a Target Node}

The probability of eclipsing a targeted node in the a network with $ k $ nodes
in $ h $ hashes is modeled by a similar binomial distribution:

{\centering
$\Pr_{success}(h, k) = \displaystyle \sum_{i=3}^{h-1}
k^{-i}(1-\frac{1}{k})^{h-i}{h \choose i}$
\\}

\begin{table}[hbt!]
\begin{center}
\begin{tabular}{l l r}
h & i & $\Pr_{success}{h,i}$\\
\hline  100 & 100 & 7.937e-02\\
\hline  100 & 500 & 1.120e-03\\
\hline  100 & 900 & 2.046e-04\\
\hline  500 & 100 & 8.766e-01\\
\hline  500 & 500 & 8.012e-02\\
\hline  500 & 900 & 1.888e-02\\
\hline  900 & 100 & 9.939e-01\\
\hline  900 & 500 & 2.693e-01\\
\hline  900 & 900 & 8.020e-02\\
\end{tabular}
\end{center}
\end{table}

Code:
\begin{lstlisting}
def fac(k): return 1 if k==0 else k * fac(k-1)
def choose(h,k): return fac(h) / fac(k) / fac(h-k)
def bin(i,h,k): return choose(h,i) * k ** -i * (1-(1.0/k)) ** (h-i)
def prob_succ(h,k): return sum([bin(i,h,k) for i in range(3,h)])
\end{lstlisting}

\subsection{Beach Size}

As the number of shards on the network grows, it becomes progressively more
difficult to locate a given file without prior knowledge of the locations of its
shards. This implies that even should all farmers become faithless, file privacy
is largely preserved.

The probability of locating a targeted file consisting of $ k $ shards by $ n $
random draws from a network containing $ N $ shards is modeled as a
hypergeometric distribution with $ K = k $:

{\centering
$Pr_{Success}(N,k,n) = \displaystyle \frac{{N-k \choose n-k}}{{N \choose n}}$
\\}

\begin{table}[hbt!]
\begin{center}
\begin{tabular}{r r l r}
N & k & n & $\Pr_{success}{N,k,n}$\\
\hline 100 & 10 & 10  & 5.777e-14\\
\hline 100 & 10 & 50  & 5.934e-04\\
\hline 100 & 10 & 90  & 3.305e-01\\
\hline 100 & 50 & 50  & 9.912e-30\\
\hline 100 & 50 & 90  & 5.493e-04\\
\hline 500 & 50 & 200 & 1.961e-22\\
\hline 500 & 50 & 400 & 7.361e-06\\
\hline 900 & 10 & 200 & 2.457e-07\\
\hline 900 & 10 & 400 & 2.823e-04\\
\hline 900 & 10 & 800 & 3.060e-01\\
\hline 900 & 50 & 200 & 1.072e-35\\
\hline 900 & 50 & 400 & 4.023e-19\\
\hline 900 & 50 & 800 & 2.320e-03\\
\end{tabular}
\end{center}
\end{table}

Code:
\begin{lstlisting}
def fac(k): return 1 if k==0 else k * fac(k-1)
def choose(h,k): return fac(h) / fac(k) / fac(h-k)
def hyp(N,k,n): return choose(N-k,n-k) / float(choose(N,n))
def prob_success(N,k,n): return hyp(N,k,n)
\end{lstlisting}

\subsection{Partial Audit Confidence Levels}

Farmers attempting to game the system may rely on data owners to issue partial
audits. Partial audits allow false positives, where the data appears intact, but
in fact has been modified. Data owners may account for this by ascribing
confidence values to each partial audit, based on the likelihood of a false
positive. Partial audit results then update prior confidence of availability.
Data owners may adjust audit parameters to provide desired confidence levels.

The probability of a false positive on a parital audit of $ n $ bytes of an $ N
$ byte shard, with $ K $ bytes modified adversarially by the farmer is a
hypergeometric distribution with $ k = 0 $:

{\centering
$Pr_{false positive}(N,K,n) = \displaystyle \frac{{N-K \choose n}} {{N \choose
n}}$
\\}

\begin{table}[hbt!]
\begin{center}
\begin{tabular}{r r l r}
N & K & n & $\Pr_{falsepositive}{N,K,n}$\\
\hline 8192 & 512  & 512 & 1.466e-15\\
\hline 8192 & 1024 & 512 & 1.867e-31\\
\hline 8192 & 2048 & 512 & 3.989e-67\\
\hline 8192 & 3072 & 512 & 1.228e-109\\
\hline 8192 & 4096 & 512 & 2.952e-162\\
\end{tabular}
\end{center}
\end{table}

Code:
\begin{lstlisting}
def fac(k): return 1 if k==0 else k * fac(k-1)
def choose(h,k): return fac(h) / fac(k) / fac(h-k)
def hyp(N,K,n): return float(choose(N-K, n) / choose(N,n)
def prob_false_pos(N,K,n): return hyp(N,K,n)
\end{lstlisting}

As demonstrated, the chance of false positives on even small partial audits
becomes vanishingly small. Farmers failing audits risk losing payouts from
current contracts, as well as potential future contracts as a result of failed
audits. Dropping 10\% of a shard virtually guarantees a loss greater than 10\%
of the contract value. Thus it stands to reason that partially deleting shards
to increase perceived storage capcity is not a viable economic strategy.

\newpage
\appendix

\section{Reed-Solomon}

\todo{}

\section{Kademlia}

\todo{}

\section{S/Kademlia}

\todo{}

\section{Macaroons}

\todo{}

\newpage
\bibliographystyle{unsrt}
\begingroup
  \raggedright
  \bibliography{biblio}
\endgroup

\end{document}
